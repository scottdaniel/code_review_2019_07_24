\documentclass[]{article}
\usepackage{lmodern}
\usepackage{amssymb,amsmath}
\usepackage{ifxetex,ifluatex}
\usepackage{fixltx2e} % provides \textsubscript
\ifnum 0\ifxetex 1\fi\ifluatex 1\fi=0 % if pdftex
  \usepackage[T1]{fontenc}
  \usepackage[utf8]{inputenc}
\else % if luatex or xelatex
  \ifxetex
    \usepackage{mathspec}
  \else
    \usepackage{fontspec}
  \fi
  \defaultfontfeatures{Ligatures=TeX,Scale=MatchLowercase}
\fi
% use upquote if available, for straight quotes in verbatim environments
\IfFileExists{upquote.sty}{\usepackage{upquote}}{}
% use microtype if available
\IfFileExists{microtype.sty}{%
\usepackage{microtype}
\UseMicrotypeSet[protrusion]{basicmath} % disable protrusion for tt fonts
}{}
\usepackage{hyperref}
\hypersetup{unicode=true,
            pdftitle={Recipes demo for Code review},
            pdfauthor={PennCHOP microbiome program, Scott Garrett Daniel},
            pdfborder={0 0 0},
            breaklinks=true}
\urlstyle{same}  % don't use monospace font for urls
\usepackage{color}
\usepackage{fancyvrb}
\newcommand{\VerbBar}{|}
\newcommand{\VERB}{\Verb[commandchars=\\\{\}]}
\DefineVerbatimEnvironment{Highlighting}{Verbatim}{commandchars=\\\{\}}
% Add ',fontsize=\small' for more characters per line
\usepackage{framed}
\definecolor{shadecolor}{RGB}{248,248,248}
\newenvironment{Shaded}{\begin{snugshade}}{\end{snugshade}}
\newcommand{\KeywordTok}[1]{\textcolor[rgb]{0.13,0.29,0.53}{\textbf{#1}}}
\newcommand{\DataTypeTok}[1]{\textcolor[rgb]{0.13,0.29,0.53}{#1}}
\newcommand{\DecValTok}[1]{\textcolor[rgb]{0.00,0.00,0.81}{#1}}
\newcommand{\BaseNTok}[1]{\textcolor[rgb]{0.00,0.00,0.81}{#1}}
\newcommand{\FloatTok}[1]{\textcolor[rgb]{0.00,0.00,0.81}{#1}}
\newcommand{\ConstantTok}[1]{\textcolor[rgb]{0.00,0.00,0.00}{#1}}
\newcommand{\CharTok}[1]{\textcolor[rgb]{0.31,0.60,0.02}{#1}}
\newcommand{\SpecialCharTok}[1]{\textcolor[rgb]{0.00,0.00,0.00}{#1}}
\newcommand{\StringTok}[1]{\textcolor[rgb]{0.31,0.60,0.02}{#1}}
\newcommand{\VerbatimStringTok}[1]{\textcolor[rgb]{0.31,0.60,0.02}{#1}}
\newcommand{\SpecialStringTok}[1]{\textcolor[rgb]{0.31,0.60,0.02}{#1}}
\newcommand{\ImportTok}[1]{#1}
\newcommand{\CommentTok}[1]{\textcolor[rgb]{0.56,0.35,0.01}{\textit{#1}}}
\newcommand{\DocumentationTok}[1]{\textcolor[rgb]{0.56,0.35,0.01}{\textbf{\textit{#1}}}}
\newcommand{\AnnotationTok}[1]{\textcolor[rgb]{0.56,0.35,0.01}{\textbf{\textit{#1}}}}
\newcommand{\CommentVarTok}[1]{\textcolor[rgb]{0.56,0.35,0.01}{\textbf{\textit{#1}}}}
\newcommand{\OtherTok}[1]{\textcolor[rgb]{0.56,0.35,0.01}{#1}}
\newcommand{\FunctionTok}[1]{\textcolor[rgb]{0.00,0.00,0.00}{#1}}
\newcommand{\VariableTok}[1]{\textcolor[rgb]{0.00,0.00,0.00}{#1}}
\newcommand{\ControlFlowTok}[1]{\textcolor[rgb]{0.13,0.29,0.53}{\textbf{#1}}}
\newcommand{\OperatorTok}[1]{\textcolor[rgb]{0.81,0.36,0.00}{\textbf{#1}}}
\newcommand{\BuiltInTok}[1]{#1}
\newcommand{\ExtensionTok}[1]{#1}
\newcommand{\PreprocessorTok}[1]{\textcolor[rgb]{0.56,0.35,0.01}{\textit{#1}}}
\newcommand{\AttributeTok}[1]{\textcolor[rgb]{0.77,0.63,0.00}{#1}}
\newcommand{\RegionMarkerTok}[1]{#1}
\newcommand{\InformationTok}[1]{\textcolor[rgb]{0.56,0.35,0.01}{\textbf{\textit{#1}}}}
\newcommand{\WarningTok}[1]{\textcolor[rgb]{0.56,0.35,0.01}{\textbf{\textit{#1}}}}
\newcommand{\AlertTok}[1]{\textcolor[rgb]{0.94,0.16,0.16}{#1}}
\newcommand{\ErrorTok}[1]{\textcolor[rgb]{0.64,0.00,0.00}{\textbf{#1}}}
\newcommand{\NormalTok}[1]{#1}
\usepackage{graphicx,grffile}
\makeatletter
\def\maxwidth{\ifdim\Gin@nat@width>\linewidth\linewidth\else\Gin@nat@width\fi}
\def\maxheight{\ifdim\Gin@nat@height>\textheight\textheight\else\Gin@nat@height\fi}
\makeatother
% Scale images if necessary, so that they will not overflow the page
% margins by default, and it is still possible to overwrite the defaults
% using explicit options in \includegraphics[width, height, ...]{}
\setkeys{Gin}{width=\maxwidth,height=\maxheight,keepaspectratio}
\IfFileExists{parskip.sty}{%
\usepackage{parskip}
}{% else
\setlength{\parindent}{0pt}
\setlength{\parskip}{6pt plus 2pt minus 1pt}
}
\setlength{\emergencystretch}{3em}  % prevent overfull lines
\providecommand{\tightlist}{%
  \setlength{\itemsep}{0pt}\setlength{\parskip}{0pt}}
\setcounter{secnumdepth}{0}
% Redefines (sub)paragraphs to behave more like sections
\ifx\paragraph\undefined\else
\let\oldparagraph\paragraph
\renewcommand{\paragraph}[1]{\oldparagraph{#1}\mbox{}}
\fi
\ifx\subparagraph\undefined\else
\let\oldsubparagraph\subparagraph
\renewcommand{\subparagraph}[1]{\oldsubparagraph{#1}\mbox{}}
\fi

%%% Use protect on footnotes to avoid problems with footnotes in titles
\let\rmarkdownfootnote\footnote%
\def\footnote{\protect\rmarkdownfootnote}


  \title{Recipes demo for Code review}
    \author{PennCHOP microbiome program, Scott Garrett Daniel}
      \date{\today}

\usepackage{pdflscape}
\usepackage{lscape}
\usepackage{tikz}
\usepackage{tikz-cd}
\usepackage{color}
\usepackage[colorinlistoftodos]{todonotes}

\newcommand{\blandscape}{\begin{landscape}}
\newcommand{\elandscape}{\end{landscape}}

\newcommand{\bscript}{\begin{scriptsize}}
\newcommand{\escript}{\end{scriptsize}}

\newcommand{\stopdocument}{\end{document}} % \newcommand cannot define that starts with \end... so use \stopdocument instead of \enddocument

\def\red{\textcolor{red}}
\def\green{\textcolor{green}}
\def\blue{\textcolor{blue}}

\begin{document}
\maketitle

{
\setcounter{tocdepth}{3}
}
\includegraphics{logo_blk.png}\\
\tableofcontents

\begin{Shaded}
\begin{Highlighting}[]
\NormalTok{### ================}
\NormalTok{###   R packages}
\NormalTok{### ================}
\CommentTok{#This package will also help us more easily manipulate our data}
\KeywordTok{library}\NormalTok{(tidyverse)}
\end{Highlighting}
\end{Shaded}

\begin{Shaded}
\begin{Highlighting}[]
\NormalTok{### ================}
\NormalTok{###   User defined functions}
\NormalTok{### ================}
\end{Highlighting}
\end{Shaded}

\begin{Shaded}
\begin{Highlighting}[]
\NormalTok{### ===========================}
\NormalTok{###   define constants}
\NormalTok{### ===========================}

\NormalTok{### setwd}
\CommentTok{#fill in your project dir}
\NormalTok{root_dir =}\StringTok{ ""}

\NormalTok{### file path}
\NormalTok{replace_me <-}\StringTok{ }\KeywordTok{file.path}\NormalTok{(root_dir, }\StringTok{""}\NormalTok{)}

\NormalTok{### metadata}
\NormalTok{metadata_fp <-}\StringTok{ }\KeywordTok{file.path}\NormalTok{(root_dir, }\StringTok{""}\NormalTok{ )}
\end{Highlighting}
\end{Shaded}

Based on:
\url{https://www.r-bloggers.com/how-to-use-recipes-package-from-tidymodels-for-one-hot-encoding-\%f0\%9f\%9b\%a0/}

Since once of the best way to learn, is to explain, I want to share with
you this quick introduction to recipes package, from the tidymodels
family. It can help us to automatize some data preparation tasks.

The overview is:

How to create a recipe How to add a step How to do the prep Getting the
data with juice! Apply the prep to new data What is the difference
between bake and juice? Dealing with new values in recipes (step\_novel)
Since I'm new to this package, if you have something to add just put in
the comments

Introduction If you are new to R or you do a 1-time analysis, you could
not see the main advantage of this, which is -in my opinion- to have
most of the data preparation steps in one place. This way is easier to
split between dev and prod.

Dev: The stage in which we create the model Prod: The moment in which we
run the model with new data The other big advantage is it follows the
tidy philosophy, so many things will be familiar.

How to use recipes for one hot encoding It is focused on one hot
encoding, but many other functions like scaling, applying PCA and others
can be performed.

But first, what is one hot encoding?

It's a data preparation technique to convert all the categorical
variables into numerical, by assigning a value of 1 when the row belongs
to the category. If the variable has 100 unique values, the final result
will contain 100 columns.

That's why it is a good practice to reduce the cardinality of the
variable before continuing Learn more about it in the High Cardinality
Variable in Predictive Modeling from the Data Science Live Book.

Let's start the example with recipes!

1st -- How to create a recipe

\begin{Shaded}
\begin{Highlighting}[]
\KeywordTok{library}\NormalTok{(recipes)}
\KeywordTok{library}\NormalTok{(tidyverse)}

\KeywordTok{set.seed}\NormalTok{(}\FloatTok{3.1415}\NormalTok{)}
\NormalTok{iris_tr=}\KeywordTok{sample_frac}\NormalTok{(iris, }\DataTypeTok{size =} \FloatTok{0.7}\NormalTok{)}

\NormalTok{rec =}\StringTok{ }\KeywordTok{recipe}\NormalTok{( }\OperatorTok{~}\StringTok{ }\NormalTok{., }\DataTypeTok{data =}\NormalTok{ iris_tr)}

\NormalTok{rec}
\end{Highlighting}
\end{Shaded}

\begin{verbatim}
## Data Recipe
## 
## Inputs:
## 
##       role #variables
##  predictor          5
\end{verbatim}

\begin{Shaded}
\begin{Highlighting}[]
\KeywordTok{summary}\NormalTok{(rec)}
\end{Highlighting}
\end{Shaded}

\begin{verbatim}
## # A tibble: 5 x 4
##   variable     type    role      source  
##   <chr>        <chr>   <chr>     <chr>   
## 1 Sepal.Length numeric predictor original
## 2 Sepal.Width  numeric predictor original
## 3 Petal.Length numeric predictor original
## 4 Petal.Width  numeric predictor original
## 5 Species      nominal predictor original
\end{verbatim}

The formula \textasciitilde{} ., specifies that all the variables are
predictors (with no outcomes).

Please note now we have two different data types, numeric and nominal
(not factor nor character).

2nd -- How to add a step Now we add the step to create the dummy
variables, or the one hot encoding, which can be seen as the same.

When we do the one hot encoding (one\_hot = T), all the levels will be
present in the final result. Conversely, when we create the dummy
variables, we could have all of the variables, or one less (to avoid the
multi-correlation issue).

\begin{Shaded}
\begin{Highlighting}[]
\NormalTok{rec_}\DecValTok{2}\NormalTok{ =}\StringTok{ }\NormalTok{rec }\OperatorTok\StringTok{ }\KeywordTok{step_dummy}\NormalTok{(Species, }\DataTypeTok{one_hot =}\NormalTok{ T)}

\NormalTok{rec_}\DecValTok{2}
\end{Highlighting}
\end{Shaded}

\begin{verbatim}
## Data Recipe
## 
## Inputs:
## 
##       role #variables
##  predictor          5
## 
## Operations:
## 
## Dummy variables from Species
\end{verbatim}

Now we see the dummy step.

3rd -- How to do the prep prep is like putting all the ingredients
together, but we didn't cook yet!

It generates the metadata to do the data preparation.

As we can see here:

\begin{Shaded}
\begin{Highlighting}[]
\NormalTok{d_prep=rec_}\DecValTok{2} \OperatorTok\StringTok{ }\KeywordTok{prep}\NormalTok{(}\DataTypeTok{training =}\NormalTok{ iris_tr, }\DataTypeTok{retain =}\NormalTok{ T)}

\NormalTok{d_prep}
\end{Highlighting}
\end{Shaded}

\begin{verbatim}
## Data Recipe
## 
## Inputs:
## 
##       role #variables
##  predictor          5
## 
## Training data contained 105 data points and no missing data.
## 
## Operations:
## 
## Dummy variables from Species [trained]
\end{verbatim}

Note we are in the ``training'' or dev stage. That's why we see the
parameter training.

We will see retain = T in the next step.

Checking:

\begin{Shaded}
\begin{Highlighting}[]
\KeywordTok{summary}\NormalTok{(d_prep)}
\end{Highlighting}
\end{Shaded}

\begin{verbatim}
## # A tibble: 7 x 4
##   variable           type    role      source  
##   <chr>              <chr>   <chr>     <chr>   
## 1 Sepal.Length       numeric predictor original
## 2 Sepal.Width        numeric predictor original
## 3 Petal.Length       numeric predictor original
## 4 Petal.Width        numeric predictor original
## 5 Species_setosa     numeric predictor derived 
## 6 Species_versicolor numeric predictor derived 
## 7 Species_virginica  numeric predictor derived
\end{verbatim}

Voila! We have the 3-new derived columns (one hot), and it removed the
original Species.

4th -- Getting the data with juice! Using juice function:

\begin{Shaded}
\begin{Highlighting}[]
\NormalTok{d2=}\KeywordTok{juice}\NormalTok{(d_prep)}

\KeywordTok{head}\NormalTok{(d2)}
\end{Highlighting}
\end{Shaded}

\begin{verbatim}
## # A tibble: 6 x 7
##   Sepal.Length Sepal.Width Petal.Length Petal.Width Species_setosa
##          <dbl>       <dbl>        <dbl>       <dbl>          <dbl>
## 1          5           3            1.6         0.2              1
## 2          6.9         3.2          5.7         2.3              0
## 3          6.3         3.3          4.7         1.6              0
## 4          5.3         3.7          1.5         0.2              1
## 5          6.3         2.3          4.4         1.3              0
## 6          6.7         3            5.2         2.3              0
## # ... with 2 more variables: Species_versicolor <dbl>,
## #   Species_virginica <dbl>
\end{verbatim}

juice worked because we retained the training data in the 3rd step
(retain = T). Otherwise it would have returned:

Error: Use retain = TRUE in prep to be able to extract the training set

5th -- Apply the prep to new data Now imagine we have new data as
follows:

\begin{Shaded}
\begin{Highlighting}[]
\NormalTok{iris_new=}\KeywordTok{sample_n}\NormalTok{(iris, }\DataTypeTok{size =} \DecValTok{5}\NormalTok{) }\CommentTok{# taking 5 random rows}

\NormalTok{d_baked=}\KeywordTok{bake}\NormalTok{(d_prep, }\DataTypeTok{new_data =}\NormalTok{ iris_new)}

\NormalTok{d_baked}
\end{Highlighting}
\end{Shaded}

\begin{verbatim}
## # A tibble: 5 x 7
##   Sepal.Length Sepal.Width Petal.Length Petal.Width Species_setosa
##          <dbl>       <dbl>        <dbl>       <dbl>          <dbl>
## 1          5.3         3.7          1.5         0.2              1
## 2          4.8         3.4          1.6         0.2              1
## 3          4.6         3.6          1           0.2              1
## 4          6.5         3            5.2         2                0
## 5          6.3         2.8          5.1         1.5              0
## # ... with 2 more variables: Species_versicolor <dbl>,
## #   Species_virginica <dbl>
\end{verbatim}

It worked!

bake receives the prep object (d\_prep) and it applies to the newdata
(iris\_new)

What is the difference between bake and juice? From this perspective
given the training data, following data frames are the same:

\begin{Shaded}
\begin{Highlighting}[]
\NormalTok{d_tr_}\DecValTok{1}\NormalTok{=}\KeywordTok{bake}\NormalTok{(d_prep, }\DataTypeTok{new_data =}\NormalTok{ iris_tr)}
\NormalTok{d_tr_}\DecValTok{2}\NormalTok{=d2=}\KeywordTok{juice}\NormalTok{(d_prep) }\CommentTok{# with retain=T}

\KeywordTok{identical}\NormalTok{(d_tr_}\DecValTok{1}\NormalTok{, d_tr_}\DecValTok{2}\NormalTok{)}
\end{Highlighting}
\end{Shaded}

\begin{verbatim}
## [1] TRUE
\end{verbatim}

Dealing with new values in recipes Simulate a new value:

\begin{Shaded}
\begin{Highlighting}[]
\NormalTok{new_row=iris[}\DecValTok{1}\NormalTok{,] }\OperatorTok\StringTok{ }\KeywordTok{mutate}\NormalTok{(}\DataTypeTok{Species=}\KeywordTok{as.character}\NormalTok{(Species))}
\NormalTok{new_row[}\DecValTok{1}\NormalTok{, }\StringTok{"Species"}\NormalTok{]=}\StringTok{"i will break your code"}

\NormalTok{new_row}
\end{Highlighting}
\end{Shaded}

\begin{verbatim}
##   Sepal.Length Sepal.Width Petal.Length Petal.Width                Species
## 1          5.1         3.5          1.4         0.2 i will break your code
\end{verbatim}

We use bake to convert the new data set:

\begin{Shaded}
\begin{Highlighting}[]
\NormalTok{d2_b=}\KeywordTok{bake}\NormalTok{(d_prep, }\DataTypeTok{new_data =}\NormalTok{ new_row)}
\end{Highlighting}
\end{Shaded}

The solution! Use step\_novel (Thanks to Max Kuhn)

When we do the prep, we have to add step\_novel. So any new value will
be assigned to the \_new category.

We will start right from the beginning:

\begin{Shaded}
\begin{Highlighting}[]
\NormalTok{rec_2_bis =}\StringTok{ }\KeywordTok{recipe}\NormalTok{( }\OperatorTok{~}\StringTok{ }\NormalTok{., }\DataTypeTok{data =}\NormalTok{ iris_tr) }\OperatorTok\StringTok{ }
\StringTok{  }\KeywordTok{step_novel}\NormalTok{(Species) }\OperatorTok\StringTok{ }
\StringTok{  }\KeywordTok{step_dummy}\NormalTok{(Species, }\DataTypeTok{one_hot =}\NormalTok{ T)}

\NormalTok{prep_bis =}\StringTok{ }\KeywordTok{prep}\NormalTok{(rec_2_bis, }\DataTypeTok{training =}\NormalTok{ iris_tr)}
\end{Highlighting}
\end{Shaded}

Get to final data, and check it:

\begin{Shaded}
\begin{Highlighting}[]
\NormalTok{processed =}\StringTok{ }\KeywordTok{bake}\NormalTok{(prep_bis, iris_tr)}

\NormalTok{funModeling}\OperatorTok{::}\KeywordTok{df_status}\NormalTok{(processed)}
\end{Highlighting}
\end{Shaded}

\begin{verbatim}
##             variable q_zeros p_zeros q_na p_na q_inf p_inf    type unique
## 1       Sepal.Length       0    0.00    0    0     0     0 numeric     32
## 2        Sepal.Width       0    0.00    0    0     0     0 numeric     23
## 3       Petal.Length       0    0.00    0    0     0     0 numeric     42
## 4        Petal.Width       0    0.00    0    0     0     0 numeric     20
## 5     Species_setosa      68   64.76    0    0     0     0 numeric      2
## 6 Species_versicolor      69   65.71    0    0     0     0 numeric      2
## 7  Species_virginica      73   69.52    0    0     0     0 numeric      2
## 8        Species_new     105  100.00    0    0     0     0 numeric      1
\end{verbatim}

Please note that Species\_new has been automatically created (with
zeros).

This ensures it runs well once in production.

Now let's see what happen when we have the new value:

\begin{Shaded}
\begin{Highlighting}[]
\NormalTok{new_row_}\DecValTok{2}\NormalTok{=}\KeywordTok{bake}\NormalTok{(prep_bis, }\DataTypeTok{new_data =}\NormalTok{ new_row)}

\NormalTok{new_row_}\DecValTok{2} \OperatorTok\StringTok{ }\KeywordTok{select}\NormalTok{(Species_new)}
\end{Highlighting}
\end{Shaded}

\begin{verbatim}
## # A tibble: 1 x 1
##   Species_new
##         <dbl>
## 1           1
\end{verbatim}

It works!

Conclusions The recipes package seems to be a good way to standardize
certain data preparation tasks. Probably one of the strongest points in
R, alongside the dplyr package.

Take care of the data pipeline, it is what interviewers will ask you
for.


\end{document}
